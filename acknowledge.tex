The ultimate success of the work presented in this thesis would not have been possible without the continued time and support of numerous individuals I've encounters throughout my life and career.

First, I would like to thank my thesis advisor Edo Berger. It has been a pleasure to work with him during the past five years on such an exciting and novel project. Without his initial push into this field I never would have started this work. I've always appreciated Edo's ability to strike a balance between letting me work on my own and pushing me beyond my limits when the project demanded it. I know his guidance has helped me be not just a better scientist but a better collaborator and member of the community.

Of course, my success in graduate school would not have been possible without the strong foundation I developed in undergraduate. During my time at University of Maryland - College Park, I was fortunate enough to work with Chris Reynolds for four years. His support and knowledge were crucial in my early development as a researcher. I am also thankful to Cole Miller for always providing guidance and forcing me to think about things in new ways.

The Astronomy Department at Harvard has been a phenomenal place to work and there are numerous members of the faculty and staff that I would like to thank. First, I am grateful for the support I've gotten from my committee members: Josh E. Grindlay (Chair), Daniel J. Eisenstein, and Brian D. Metzger (External). I am always appreciative of their advice and feedback. I want to thank other members of the faculty for providing valueable support. Lastly, the department would not be able to run without the tireless efforts of the administrative staff including Lisa Catella, Peg Herlihy, and Robb Scholten.

I am also thankful to the amazing research colleagues and collaborations I have developed during my time in graduate school. First, I am fortunate enough to be involved in a great research group here at Harvard -- The Berger Cosmic Transients Lab. The combined knowledge and talent of everyone involved, both past and present, has been essential in my development as a well-rounded scientist.

I have also been fortunate enough to build collaborations outside of my immediate group at Harvard. First, I would like to thank Brian Metzger, for helping me develop the ``theorist" side of my brain and being an all around fantastic collaborator. I am also thankful to James Guillochon, Dan Holz, Dan Kasen, and Daniel Scolnic for the many many hours they've invested in the development of my work. Looking back over the past five years, I am truly humbled by the amazing people I've been fortunate enough to work with.

To my friends and loved ones, both near and far, who provided much needed outlets from research.

Lastly, to my family, who always enoucouraged me to pursue my interests, no matter what. While this freedom in life ultimately resulted in a circuitous path to this point, I wouldn't be who I am otherwise.

%The Berger Time-Domain Group at Harvard is supported in part by the NSF through grants AST-1411763 and AST-1714498, and by NASA through grants NNX15AE50G and NNX16AC22G. P.S.C. is grateful for support provided by the NSF through the Graduate Research Fellowship Program, grant DGE1144152. V.A.V. acknowledges support by the National Science Foundation through a Graduate Research Fellowship. EB acknowledges financial support from the European Research Council (ERC-StG-335936, CLUSTERS). DK \& EQ were funded in part by the Gordon and Betty Moore Foundation through Grant GBMF5076. DAB is supposed by NSF award PHY-1707954.
%
%The UCSC group is supported in part by NSF grant AST--1518052, the Gordon \& Betty Moore Foundation, the Heising-Simons Foundation, generous donations from many individuals through a UCSC Giving Day grant, and from fellowships from the Alfred P.\ Sloan Foundation and the David and Lucile Packard Foundation to R.J.F. R.J.F.\ gratefully acknowledges support from NSF grant AST--1518052 and the Alfred P.\ Sloan Foundation. D.E.H.\ was supported by NSF CAREER grant PHY-1151836. He also acknowledges support from the Kavli Institute for Cosmological Physics at the University of Chicago through NSF grant PHY-1125897 as well as an endowment from the Kavli Foundation.
%
%R.J.F.\ thanks the University of Copenhagen, DARK Cosmology Centre, and the Niels Bohr International Academy for hosting during the discovery of GW170817/SSS17a, where he was participating in the Kavli Summer Program in Astrophysics, ``Astrophysics with gravitational wave detections."  This program was supported by the Kavli Foundation, Danish National Research Foundation, the Niels Bohr International Academy, and the DARK Cosmology Centre.
%
%This thesis uses services or data provided by the NOAO Science Archive. NOAO is operated by the Association of Universities for Research in Astronomy (AURA), Inc. under a cooperative agreement with the National Science Foundation. The computations in this thesis were run on the Odyssey cluster supported by the FAS Division of Science, Research Computing Group at Harvard University. This thesis has made use of the NASA/IPAC Extragalactic Database (NED) which is operated by the Jet Propulsion Laboratory, California Institute of Technology, under contract with the National Aeronautics and Space Administration. Some of the results in this thesis have been derived using the HEALPix package \citep{Gorski+05}.
%
%The work presented in Chapter 2 relied upon Ryan Chornock, Maria Drout, Wen-fai Fong, Ryan Foley, Daniel Kasen, Brian Metzger, Armin Rest, and Ken Shen for helpful discussions and providing model data during the course of the analysis.
%
%This thesis has made use of data from the European Space Agency (ESA) mission \gaia\ (\url{https://www.cosmos.esa.int/gaia}), processed by the \gaia Data Processing and Analysis Consortium (DPAC, \url{https://www.cosmos.esa.int/web/gaia/dpac/consortium}). Funding for the DPAC has been provided by national institutions, in particular the institutions participating in the \gaia Multilateral Agreement. This thesis makes use of data products from the Two Micron All Sky Survey, which is a joint project of the University of Massachusetts and the Infrared Processing and Analysis Center/California Institute of Technology, funded by the National Aeronautics and Space Administration and the National Science Foundation. This thesis makes use of data products from the Wide-field Infrared Survey Explorer, which is a joint project of the University of California, Los Angeles, and the Jet Propulsion Laboratory/California Institute of Technology, funded by the National Aeronautics and Space Administration.
%
%The light curve data for PS15cdi used in Chapter 4 were obtained from The Open Supernova
%Catalog \citep{Guillochon+17a}.
%
%This thesis used data obtained with the Dark Energy Camera (DECam), which was constructed by the Dark Energy Survey (DES) collaboration. Funding for the DES Projects has been provided by the U.S. Department of Energy, the U.S. National Science Foundation, the Ministry of Science and Education of Spain, the Science and Technology FacilitiesCouncil of the United Kingdom, the Higher Education Funding Council for England, the National Center for Supercomputing Applications at the University of Illinois at Urbana-Champaign, the Kavli Institute of Cosmological Physics at the University of Chicago, the Center for Cosmology and Astro-Particle Physics at the Ohio State University, the Mitchell Institute for Fundamental Physics and Astronomy at Texas A\&M University, Financiadora de Estudos e Projetos, Funda{\c c}{\~a}o Carlos Chagas Filho de Amparo {\`a} Pesquisa do Estado do Rio de Janeiro, Conselho Nacional de Desenvolvimento Cient{\'i}fico e Tecnol{\'o}gico and the Minist{\'e}rio da Ci{\^e}ncia, Tecnologia e Inova{\c c}{\~a}o, the Deutsche Forschungsgemeinschaft, and the Collaborating Institutions in the Dark Energy Survey.
%
%The Collaborating Institutions are Argonne National Laboratory, the University of California at Santa Cruz, the University of Cambridge, Centro de Investigaciones Energ{\'e}ticas, Medioambientales y Tecnol{\'o}gicas-Madrid, the University of Chicago, University College London, the DES-Brazil Consortium, the University of Edinburgh, the Eidgen{\"o}ssische Technische Hochschule (ETH) Z{\"u}rich,  Fermi National Accelerator Laboratory, the University of Illinois at Urbana-Champaign, the Institut de Ci{\`e}ncies de l'Espai (IEEC/CSIC), the Institut de F{\'i}sica d'Altes Energies, Lawrence Berkeley National Laboratory, the Ludwig-Maximilians Universit{\"a}t M{\"u}nchen and the associated Excellence Cluster Universe, the University of Michigan, the National Optical Astronomy Observatory, the University of Nottingham, The Ohio State University, the OzDES Membership Consortium, the University of Pennsylvania, the University of Portsmouth, SLAC National Accelerator Laboratory, Stanford University, the University of Sussex, and Texas A\&M University.
%
%The DES Data Management System is supported by the NSF under Grant Number AST-1138766. The DES participants from Spanish institutions are partially supported by MINECO under grants AYA2012-39559, ESP2013-48274, FPA2013-47986, and Centro de Excelencia Severo Ochoa SEV-2012-0234. Research leading to these results has received funding from the ERC under the EU's 7$^{\rm th}$ Framework Programme including grants ERC 240672, 291329 and 306478.
%
%This thesis is based in part on observations at Cerro Tololo Inter-American Observatory, National Optical Astronomy Observatory (NOAO Prop. ID: 2017B-0110, PI: E. Berger), which is operated by the Association of Universities for Research in Astronomy (AURA) under a cooperative agreement with the National Science Foundation
%
%This thesis is based in part on observations obtained at the Gemini Observatory (Program IDs GS-2017B-Q-8 and GS-2017B-DD-4; PI: Chornock), which is operated by the Association of Universities for Research in Astronomy, Inc., under a cooperative agreement with the NSF on behalf of the Gemini partnership: the National Science Foundation (United States), the National Research Council (Canada), CONICYT (Chile), Ministerio de Ciencia, Tecnolog\'{i}a e Innovaci\'{o}n Productiva (Argentina), and Minist\'{e}rio da Ci\^{e}ncia, Tecnologia e Inova\c{c}\~{a}o (Brazil).
%
%We are grateful for the heroic efforts of the entire staff at Gemini-South to continue obtaining observations of GW170817 in evening twilight at high airmass as the object was setting.
