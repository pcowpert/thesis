%Chapter 2 Stuff
We thank Ryan Chornock, Maria Drout, Wen-fai Fong, Ryan Foley, Daniel Kasen, Brian Metzger, Armin Rest, and Ken Shen for helpful discussions and providing model data during the course of this analysis. P.S.C. is grateful for support provided by the NSF through the Graduate Research Fellowship Program, grant DGE1144152.

%Chapter 3 Stuff
The Berger Time-Domain Group at Harvard is supported in part by the NSF through grants AST-1411763 and AST-1714498, and by NASA through grants NNX15AE50G and NNX16AC22G. P.S.C. is grateful for support provided by the NSF
through the Graduate Research Fellowship Program, grant DGE1144152. The UCSC group is supported in part by NSF grant AST--1518052, the Gordon \& Betty Moore Foundation, and from fellowships from the Alfred P.\ Sloan Foundation and the David and Lucile Packard Foundation to R.J.F.

This work has made use of data from the European Space Agency (ESA)
mission {\it Gaia} (\url{https://www.cosmos.esa.int/gaia}), processed by
the {\it Gaia} Data Processing and Analysis Consortium (DPAC,
\url{https://www.cosmos.esa.int/web/gaia/dpac/consortium}). Funding
for the DPAC has been provided by national institutions, in particular
the institutions participating in the {\it Gaia} Multilateral Agreement.
This publication makes use of data products from the Two Micron All Sky Survey,
which is a joint project of the University of Massachusetts and the Infrared Processing
and Analysis Center/California Institute of Technology, funded by the National Aeronautics
and Space Administration and the National Science Foundation. This publication makes use
of data products from the Wide-field Infrared Survey Explorer, which is a joint project of the
University of California, Los Angeles, and the Jet Propulsion Laboratory/California Institute of
Technology, funded by the National Aeronautics and Space Administration.
