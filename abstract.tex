\abstract{
The direct detection of gravitational waves from the inspiral and merger of compact object binaries by the Advanced LIGO and Virgo interferometers has ushered in an exciting new era of astronomy.  Analysis of this gravitational wave data provides fundamental insight into GR in a strong gravity regime not normally accessible and allows the detection of binary systems not visible via electromagnetic observations. However, truly maximizing the science gains from these events requires the joint detection of a coincident electromagnetic counterpart. Doing so will provide new insight into the environment and host galaxy of the merger, an accurate determination of distance and energy scales, and insight into the hydrodynamics of the merger. The most promising counterpart for this task is a ``kilonova," an optical/NIR transient powered by the radioactive decay of heavy r-process elements synthesized in the merger ejecta. In this thesis, I present a series of studies that culminate in the first joint detection of gravitational waves and electromagnetic radiation from the merger of a binary neutron star system. 

First, I present studies of observational strategies to detect kilonovae and reject contaminating false-positives from unrelated transients. Using simulated observations, I show that efficient kilonova detection requires nightly observations achieving depths of $i \approx 24$~mag and $z \approx 23$ mag, ideally starting within twelve hours of a gravitational wave trigger. Furthermore, I show that kilonovae are well separated from other unrelated transients (e.g., supernovae) on the basis of their red $i-z$ colors and shorter timescales. I confirm these results with an empirical study of contamination using data taken with the Dark Energy Camera (DECam). I show that the expected contamination rate for kilonova-like transients is low with $\mathcal{R}_{\rm red} \approx 0.16$~events per deg$^2$ at a limiting magnitude of $i \lesssim 22.5$~mag. 

Second, I present results from optical follow-up observations of gravitational wave events conducted with DECam. I discuss follow-up of GW151226, the second binary black hole merger detected by the Advanced LIGO interferometers. I show that while our DECam program did not identify an electromagnetic counterpart to this event, the presence of an errant Type II-P Supernova in these observations highlights the unique challenge faced in rejecting false-positives. I then discuss follow-up of the first binary neutron star merger detected by Advanced LIGO and Virgo, GW170817, including an independent discovery of the optical counterpart by our DECam program. I present modeling of the broadband optical/NIR photometry and show that this optical emission is consistent with expectations for a kilonova. I also show that the amount of material ejected during the merger is sufficient to suggest that binary neutron star mergers are a dominant site of cosmic r-process nucleosynthesis.
}
