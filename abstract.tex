\abstract{
The direct detection of gravitational waves from the inspiral and merger of compact object binaries by the Advanced LIGO and Virgo interferometers has ushered in an exciting new era of astronomy. Analysis of this gravitational wave data provides fundamental insight into GR in a strong gravity regime not normally accessible and allows the detection of binary systems not visible via electromagnetic observations. However, truly maximizing the science gains from these detections requires the joint detection of a coincident electromagnetic counterpart. Doing so will provde new insight into the envionment and host galaxy of the merger, an accurate determination of distance and energy scales, and insight into the hydrodynamics of the merger. The most promising counterpart for this task is the ``kilonovae," an optical/NIR transient powered by the radioactive decay of heavy $r$-process elements synthesized in the merger ejecta.

In this thesis, I present a series of studies that culminate in the first joint detection of gravitational waves and electromagnetic radiation from the merger of a binary neutron star system. First, I will present studies of how to design and execute observational strategies to detect optical counterparts of interest and reject contaminating false-positives. Using both simulated and empirical observations I show that kilonovae can be well separated from other transients on the basis of their timescales and color, and that the rate of contaminating transients will be low.

Second, I will present our efforts to follow-up gravitational wave events with the Dark Energy Camera (DECam). In particular, I will focus on our follow-up of the first binary neutron star merger detected by Advanced LIGO and Virgo, GW170817. Our DECam program was able to make an independent discovery of the optical counterpart to GW170817. I will present my modeling of the broadband optical/NIR photometry that shows this optical emission is consistent with expectations for a kilonova and that the amount of material ejected during the merger is sufficient to suggest that binary neutron star mergers can be a dominant site of cosmic $r$-process nucleosynthesis.
}
