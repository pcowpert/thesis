%UNLIKE IN A REGULAR TEX FILE, DON'T PUT ANY PREAMBLE MATERIAL HERE
\section*{Abstract}
We present preliminary studies into the development of efficient and effective strategies for the usage of LSST in optical follow-up of gravitational wave events detected by the Advanced LIGO and Virgo interferometers. Specifically, we focus on the detection and characterization of kilonovae across a wide range of ejecta properties and composition. We show that for the main LSST survey (RESULTS). In contrast, focused target-of-opportunity follow up with an investment of N minutes per night for M nights allows (OTHER RESULTS). We find that (MORE RESULTS). Something about using LSST for TOO stuff.

\clearpage
\section{Introduction}
\label{sec:ch6_intro}
The discovery of an optical counterpart associated with the the binary neutron star merger GW170817 was a watershed moment in the development of joint gravitational wave and electromagnetic (GW-EM) astronomy \citep{LIGOGW170817,LIGOMMAPaper,Arcavi+17,Coulter+17,GW170817DECam,Valenti+17}. Subsequent modeling of the optical light curves revealed behavior consistent with that of a kilonova \citep{Cowp+17,Kilpatrick+17,Tanaka+17,Villar+17b, Tanaka+18}, an optical/NIR transient expected to accompany compact object mergers involving at least one neutron star \citep[for a review see e.g.,][]{Metzger2017}. The discovery of this optical transient has opened up numerous new and exciting science possibilities. These include studies into the host galaxy \citep[NGC4993, see e.g.,][]{Blanchard+17,Cantiello+18}, constraining the neutron star equation of state \citep[see e.g.,][]{Radice+18}, and even making independent measurements of the local Hubble Constant \citep[H$_0$, see e.g.,][]{LIGOH0,Guidorzi+17}.

If we wish to build upon the success of GW170817 and push into new realms of joint GW-EM science then we must work to facilitate future detections of both gravitational wave sources {\em and} their associated electromagnetic counterparts. A key component of future searches for gravitational wave events is improving the ability of the interferometer network to localize sources on the sky. Over the next several years, as both Advanced LIGO and Virgo reach their design sensitivity, it is expected that they will be able to localize compact binary mergers to sky areas of approximately tens to hundreds of deg$^2$ \citep{LIGOLocalization,ChenHolz16}. Major improvements to sky localization will continue over the next decade as two additional interferometers, KAGRA in Japan \citep{KAGRA} and LIGO-India \citep{LIGOIndia}, join the network. In this five detector regime, compact binary mergers will be localized to just $\apx10$~deg$^2$ \citep{Fairhurst2014,ChenHolz16}.

Similarly, improving our ability to identify electromagnetic counterparts in these localizations will rely on using the next generation of optical facilities that are soon to be coming online. One such facility, the Large Synoptic Survey Telescope \citep[LSST,][]{Ivezic+09}, will be the premiere time-domain in the Southern hemisphere during the next decade. LSST boasts an 8.4-m primary mirror and a 9.6 deg$^2$ field-of-view. This powerful combination of large aperture and wide field of view makes LSST uniquely suited to the task of gravitational wave follow-up. The field-of-view is particularly well matched to the expected localization regions allowing LSST to observe a high fraction of the localization probability in just one or two pointings.

Of particular importance is the development of efficient and effective strategies for gravitational wave follow-up with LSST. This first involves understanding both the rate and properties of kilonovae detected in the LSST main survey. This was investigated by \citet{Scolnic+18} who injected model light curves of the kilonova associated GW170817 \citep{Cowp+17} into the LSST cadence. They found that LSST should detect $\apx7$ GW170817-like kilonovae per year during the ten year main survey. However, these kilonovae were detected only a few times, leading to poorly-sampled light curves. As a result, real-time identification of modeling of these kilonovae will be difficult. Furthermore, these kilonovae are detected far beyond the sensitivity horizon for LIGO, meaning that an association between these kilonovae and a gravitational wave detection is unlikely. This suggests that triggered target-of-opportunities may be a more promising approach.

Here we expand on the groundwork established by \citet{Scolnic+18} along two new avenues. First, we expand the range of kilonovae models considered in the LSST main survey. This is accomplished by considering a wider range of ejecta masses and composition to fully explore the potential range of kilonovae brightness and timescales. Second, we expand beyond the main survey cadence and explore the detectability of kilonovae in triggered target-of-opportunity follow-up observations. We explore not only the ability of such observations to detect kilonovae, but also identify and characterize their behavior. 

This work is organized as follows: In \cref{sec:ch6_models} we describe the expanded range of kilonovae models used in our simulated observations. In \cref{sec:ch6_obs} we describe our methodology for simulating LSST observations using both OpSim for the main survey and SNANA for target-of-opportunity observations In \cref{sec:ch6_analysis} we present the results of our simulated observations. Lastly, discussions and conclusions are presented in \cref{sec:ch6_conc}.

All magnitudes presented in this work are given in the AB system unless otherwise noted. Cosmological calculations are performed using the cosmological parameters $H_0 = 67.7$ km s$^{-1}$ Mpc$^{-1}$, $\Omega_M = 0.307$, and $\Omega_{\Lambda} = 0.691$ \citep{Planck2016}.

\section{Kilonova Models}
\label{sec:ch6_models}
Brief description of model construction + cite previous papers.

Describe actual models produced

Describe construction of red + blue models.

Describe Usage of GW170817

\section{Simulated Observations}
\label{sec:ch6_obs}

We consider two observing scenarios, survey and too

\subsection{OpsSim}
\label{sec:ch6_opssim}
Describe opssim

Our Setup

\subsection{SNANA/ToO}
\label{sec:ch6_snana}
Describe SNANA and our set up

Outline ToO strategies and considerations

\section{Analysis}
\label{sec:ch6_analysis}
Describe analysis goals. Detection and characterization of KN. 

\subsection{OpsSim Results}
\label{sec:ch6_opssim_results}
We have such and such LC and find this rate

Some number are detected twice

Try to fit them?

\subsection{ToO Results}
\label{sec:ch6_too_results}
From SNANA we have such and such results.

Try to fit LC returned by SNANA

Look at things like ability to measure ejecta properties as a function of time investment

\section{Discussion and Conclusion}
\label{sec:ch6_conc}
We have done some LSST stuff 

Main survey summary

ToO summary

LSST must be involved in direct GW follow-up, well matched and won't find many in main survey in a usable sense.

%NO BIB INFO