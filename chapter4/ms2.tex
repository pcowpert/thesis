\section*{Abstract}
We report the results of a Dark Energy Camera (DECam) optical follow-up of the gravitational wave (GW) event GW151226, discovered by the Advanced LIGO detectors. Our observations cover 28.8 deg$^2$ of the localization region in the $i$ and $z$ bands (containing 3\%  of the {\tt BAYESTAR} localization probability), starting 10 hours after the  event was announced and spanning four epochs at $2-24$ days after the  GW detection. We achieve $5\sigma$ point-source limiting  magnitudes of $i\approx21.7$ and $z\approx21.5$, with a scatter of $0.4$~mag, in our difference images. Given the two day delay,  we search this area for a rapidly declining optical counterpart with $\gtrsim 3\sigma$ significance steady decline between the first and final observations. We recover four sources that pass our selection criteria,  of which three are cataloged AGN. The fourth source is offset by $5.8$ arcsec from the center of a galaxy at a distance of 187 Mpc, exhibits a rapid decline by $0.5$ mag over $4$ days, and has a red color of $i-z\approx 0.3$ mag. These properties could satisfy a set of cuts designed to identify kilonovae. However, this source was detected several times, starting $94$ days prior to GW151226,   in the Pan-STARRS Survey for Transients (dubbed as PS15cdi) and is therefore unrelated to the GW event. Given its long-term  behavior, PS15cdi is likely a Type IIP supernova that  transitioned out of its plateau phase during our  observations, mimicking a kilonova-like behavior. We comment on the implications of this detection for contamination in future optical follow-up observations.

\section{Introduction}
\label{sec:ch4_intro}
The Advanced Laser Interferometer Gravitational-Wave Observatory (LIGO) is designed to detect the final inspiral and merger of compact object binaries comprised of neutron stars (NS) and/or stellar-mass black holes (BH) \citep{LIGOMainRef}.  The first LIGO observing run (designated O1) commenced on 18 September 2015 with the ability to detect binary neutron star (BNS) mergers to an average distance of $\approx 75$ Mpc, a forty-fold increase in volume relative to the previous generation of ground-based GW detectors \citep{Martynov+16}. On 2015 September 14 LIGO detected  the first GW event ever observed, GW150914 \citep{LIGOGW150914}.

The waveform of GW150914 was consistent with the inspiral, merger, and ring-down of a binary black hole (BBH) system ($36+29$ M$_\odot$;  \citealt{LIGOGW150914}) providing the first observational evidence that such systems exist and merge. While there are no robust theoretical predictions for the expected electromagnetic (EM) counterparts of such a merger, more than 20 teams conducted a wide range of follow-up observations spanning from radio to $\gamma$-rays, along with neutrino follow up \citep{LIGOGW150914FollowUp,LIGOGW150914FollowUpSupp,GW150914IceCube,Annis+16,GW150914Fermi,Evans+16,Kasliwal+16,Savchenko+16,GW150914PS1,GW150914DECam,Tavani+16}. This effort included deep optical follow-up observations by our group using DECam covering 100 deg$^2$ (corresponding to a contained probability of 38\%~(11\%) of the initial~(final) sky maps) -- making this one of the most comprehensive optical  follow-up campaigns for GW150914 \citep{Annis+16,GW150914DECam}. Our search for rapidly declining transients to limiting magnitudes of $i\approx21.5$ mag for red $(i-z=1)$ and $i\approx20.1$ mag for blue $(i-z=-1)$ events yielded no counterpart to GW150914 \citep{GW150914DECam}. One result of the broader multi-wavelength follow-up campaign is a claimed coincident detection of a weak short gamma-ray burst (SGRB) from the {\it Fermi}/GBM detector 0.4 s after the GW event \citep{GW150914Fermi}. However, this event was not detected in INTEGRAL $\gamma$-ray data \citep{Savchenko+16} and was also disputed in a re-analysis of the GBM data \citep{Greiner+16}.

A second high-significance GW event, designated GW151226, was discovered by LIGO on 2015 December 26 at 03:38:53 UT \citep{LIGOGW151226}.  This event was also due to the inspiral and merger of a BBH system, consisting  of $14.2^{+8.3}_{-3.7}$~M$_\odot$ and $7.5^{+2.3}_{-2.3}$~M$_\odot$ black holes at a luminosity  distance of $d_L = 440^{+180}_{-190}$~Mpc \citep{LIGOGW151226}. The initial localization was provided as a probability sky map via a private  GCN circular 38 hours after the GW detection \citep{SingerPrice16}. We initiated optical follow-up observations with DECam 10 hours later on 2015 December 28, and imaged a 28.8 deg$^2$ region in the $i$ and $z$ bands during several epochs. Here we report  the results of this search.

In \cref{sec:ch4_obs} we discuss the observations and data analysis procedures. In \cref{sec:ch4_analysis}  we present our search methodology for potential counterparts to GW151226,  and the results of this search. We summarize our conclusions in \cref{sec:ch4_conc}. We perform  cosmological calculations assuming $H_0 = 67.8$~km  s$^{-1}$ Mpc$^{-1}$, $\Omega_{\lambda} = 0.69$, and $\Omega_m = 0.31$ \citep{Planck2016}.  Magnitudes are reported in the AB system.

\clearpage
\section{Observations and Data Reduction}
\label{sec:ch4_obs}
GW151226 was detected on 2015 December 26 at 03:38:53 UT by a  Compact Binary Coalescence (CBC) search pipeline \citep{LIGOGW151226}. The CBC pipeline operates by matching the strain data against waveform templates and is sensitive to mergers containing NS and/or BH.  The initial sky map was generated by the {\tt BAYESTAR} algorithm and released 38 hours after the GW detection. {\tt BAYESTAR} is a Bayesian  algorithm that generates a localization sky map based on the parameter estimation from the CBC pipeline \citep{Singer+14,SingerPrice16}. The sky area contained within the initial 50\% and 90\% contours was 430 deg$^2$ and 1340 deg$^2$, respectively. A sky map generated by the {\tt LALInference} algorithm, which computes the localization using Bayesian forward-modeling  of the signal morphology \citep{Veitch+15}, was released on 2016 January 15 UT,  after our DECam observations had been concluded. The {\tt LALInference} sky map is slightly narrower than the sky map from {\tt BAYESTAR} with 50\% and 90\% contours  of 362 deg$^2$ and 1238 deg$^2$, respectively.

We initiated follow-up observations with DECam on 2015 December 28 UT, two days after the GW detection and 10 hours after distribution of the {\tt BAYESTAR} sky map. DECam is a wide-field optical imager with a 3 deg$^2$ field of view  \citep{Flaugher+15}. We imaged a 28.8 deg$^2$ region corresponding to 3\% of the sky localization probability when convolved with the initial {\tt BAYESTAR} map  and 2\% of the localization probability in the final {\tt LALInference} sky map.  The pointings and ordering of the DECam observations were determined using the automated algorithm described in \citet{GW150914DECam}. The choice  of observing fields was constrained by weather, instrument availability, and the available time to observe this sky region given its high airmass. We obtained four epochs of data with each epoch consisting of one 90 s exposure in $i$-band and two 90 s exposures in $z$-band for each of the 12 pointings. The first epoch was obtained 2--3 days after the GW event time (2015 December 28--29 UT), the second epoch was at 6 days (2016 January 1 UT), the third epoch was at 13--14 days (2016 January 8--9), and the fourth epoch was at 23--24 days (2016 January 18--19). A summary of the observations is provided in \cref{tab:ch4_datatable} and a visual representation of the sky region is shown in \Cref{fig:ch4_obs}.

\begin{deluxetable}{lrccccccc}
\singlespace
\tabletypesize{\footnotesize}
\tablecolumns{9}
\tablewidth{0pt}
\tablecaption{Summary of DECam Observations of GW151226 \label{tab:ch4_datatable}}
\tablehead{
\colhead{Visit}    &
\colhead{UT}       &
\colhead{$\Delta t$\tablenotemark{a}} & \colhead{$\langle$PSF$_i$$\rangle$}    & \colhead{$\langle$PSF$_z$$\rangle$}    & \colhead{$\langle$airmass$\rangle$}  & \colhead{$\langle$depth$_i\rangle$}   & \colhead{$\langle$depth$_z\rangle$}  & \colhead{$A_{\mathrm{eff}}$\tablenotemark{b}} \\
\colhead{ } &
\colhead{ } &
\colhead{(days)} &
\colhead{(arcsec)} &
\colhead{(arcsec)} &
\colhead{ } &
\colhead{(mag)} &
\colhead{(mag)} &
\colhead{(deg$^2$)}  }
\startdata
Epoch 1 & 2015-12-28.11 & 1.96 & 0.97 & 0.99 & 1.95 & 22.39 & 22.23 & 14.4 \\      %  6 hexes
& 2015-12-29.11 & 2.96 & 1.00 & 0.97 & 1.78 & 22.57 & 22.46 & 14.4 \\      %  6 hexes
Epoch 2 & 2016-01-01.06 & 5.91 & 0.95 & 0.90 & 1.57 & 21.37 & 21.06 & 28.8 \\      % 12 hexes
Epoch 3 & 2016-01-08.11 & 12.96 & 1.68 & 1.62 & 2.15 & 22.09 & 21.70 & 24.0 \\      % 10 hexes
& 2016-01-09.11 & 13.96 & 1.17 & 1.12 & 1.80 & 22.44 & 22.17 &  4.8 \\      %  2 hexes
Epoch 4 & 2016-01-18.03 & 22.88 & 1.21 & 1.20 & 1.48 & 22.00 & 22.01 & 12.0 \\      %  5 hexes
& 2016-01-19.01 & 23.86 & 1.29 & 1.25 & 1.71 & 21.86 & 21.90 & 16.8 \\ %  7 hexes
\enddata
\tablecomments{Summary of our DECam follow-up observations of GW151226. The PSF, airmass, and depth are the average values across all observations on that date. The reported depth corresponds to the mean  $5\sigma$ point source detection in the coadded search images.}
\tablenotetext{a}{Time elapsed between the GW trigger time and the time of the first image.}
\tablenotetext{b}{The effective area corresponds to 12 DECam pointings  taking into account that $\approx 20\%$ of the 3 deg$^2$ field of view of DECam is lost due to chip gaps (10\%), 3 dead CCDs (5\%, \citealt{Diehl+14}), and masked edge pixels (5\%).}
\end{deluxetable}

\begin{figure}[t!]
\centering
\includegraphics[width=0.75 \textwidth]{./figs/chapter4/fig1.pdf}
\caption{\singlespace Sky region covered by our DECam observations (red hexagons) relative to the 50\% and 90\% probability regions  from the {\tt BAYESTAR} (cyan contours) and {\tt LALInference} (white contours) localization of GW151226.  The background color indicates the estimated $5\sigma$  point-source limiting magnitude for a 90~s $i$-band exposure  as a function of sky position for the first night of our DECam  observations. The variation in the limiting magnitude is largely  driven by the dust extinction and airmass at that position.  The dark grey regions indicate sky positions that  were unobservable due to the telescope pointing limits.  The yellow contour  indicates the region of sky covered by the Dark Energy Survey  (DES). The total effective area for the 12 DECam pointings is 28.8  deg$^2$, corresponding to 3\%~(2\%) of the probability in the {\tt BAYESTAR}~({\tt LALInference}) sky map.}
\label{fig:ch4_obs}
\end{figure}

We processed the data using an implementation of the {\tt photpipe}  pipeline modified for DECam images. {\tt Photpipe} is a pipeline used in several time-domain surveys (e.g., SuperMACHO, ESSENCE, Pan-STARRS1; see \citealt{Rest+05,Garg+07,Miknaitis+07,Rest+14}), designed to perform single-epoch image processing including image calibration (e.g., bias subtraction, cross-talk corrections, flat-fielding), astrometric calibration, image coaddition, and photometric calibration. Additionally, {\tt photpipe} performs difference imaging using {\tt hotpants} \citep{Alard2000,Becker2015} to compute a spatially varying convolution kernel, followed by photometry on the difference images using an implementation of {\tt DoPhot} optimized for point spread function (PSF) photometry on difference images \citep{Schechter+93}. Lastly, we use {\tt photpipe} to perform initial candidate searches by specifying a required number of spatially coincident detections over a range of time. Once candidates are identified, {\tt photpipe} performs ``forced" PSF photometry on the subtracted images at the fixed coordinates of an identified candidate in each available epoch.

\clearpage
In the case of the GW151226 observations, we began with raw images acquired from the NOAO archive\footnote{\singlespace http://archive.noao.edu/} and the most recent calibration files\footnote{\singlespace http://www.ctio.noao.edu/noao/content/decam-calibration-files}. Astrometric calibration was performed relative to the Pan-STARRS1 (PS1) $3\pi$ survey and 2MASS $J$-band catalogs. The two $z$-band exposures were then coadded. Photometric calibration was performed using the PS1 $3\pi$ survey with appropriate calibrations between  PS1 and DECam magnitudes \citep{Scolnic+15}. Image subtraction was performed using observations from the final epoch as templates. The approach to candidate selection is described in \cref{sec:ch4_analysis}.

Our observations achieved average $5\sigma$ point-source limiting magnitudes of $i\approx22.2$ and $z\approx21.9$ in the coadded single-epoch search images, and $i\approx21.7$ and $z\approx21.5$ in the difference images, with an epoch-to-epoch scatter of 0.4 mag. The variability in  depth is driven by the high airmass and changes in observing conditions, particularly during the second epoch.

\section{Search for an Optical Counterpart}
\label{sec:ch4_analysis}
The primary focus of our search is a fast-fading transient.  While the merger of a BBH system is not expected to produce an EM counterpart, it is informative to consider the possibility of optical emission due to the presence of some matter in the system. As a generic example, we consider the behavior of a transient such as a short gamma-ray burst (SGRB) with a typical beaming-corrected energy of $E_j \approx 10^{49}$ erg and an opening angle of $\theta_j\approx 10^\circ$ \citep{Berger2014,Fong+15}. If viewed far off-axis $(\theta_{\rm obs}\gtrsim 4\theta_j)$ the optical emission will reach peak brightness after several days, but at the distance of GW151226 ($\approx440$ Mpc, \citealt{LIGOGW151226}), the peak brightness will be $i\approx 26$ mag (see Figure 5 of \citealt{MetzgerBerger12}), well beyond our detection limit. If the source is observed moderately off-axis or on-axis $(\theta_{\rm obs} \lesssim 2\theta_j)$, then the light curve will decline throughout our observations, roughly as $F_\nu\propto t^{-1}$, and will be detectable at $i\approx$ 21--22 mag in our first observation (see Figures 3 and 4 of \citealt{MetzgerBerger12}). We can apply a similar argument to the behavior of a more isotropic (and non-relativistic) outflow given that any material ejected in a BBH merger is likely to have a low mass and the outflow will thus become optically thin early, leading to fading optical emission. Based on this line of reasoning, we search our data for steadily declining transients.

We identify relevant candidates in the data using the following selection criteria with the forced photometry from {\tt photpipe}.  Unless otherwise noted these criteria are applied to the $i$-band  data due to the greater depth in those observations.
\begin{enumerate}
\item We require non-negative or consistent with zero  (i.e., within 2$\sigma$ of zero)  $i$- and $z$-band fluxes in the difference photometry across all epochs to eliminate any sources that re-brighten in the  fourth (template) epoch. This provides an initial sample of 602 candidates.

\item We require $\ge 5\sigma$ $i$- and $z$-band detections in the first epoch and at least one additional $\ge 5\sigma$ $i$-band detection in either of the two remaining epochs (to eliminate contamination from asteroids). This criterion leaves a sample of 98 objects.

\clearpage
\item We require a $\ge 3\sigma$ decline in flux between the first and third epochs to search for significant fading\footnote{\singlespace We note that this  criterion effectively requires the detection in the first epoch to be $\gtrsim 5\sigma$  producing an effectively shallower transient search.  \citet{GW150914DECam} quantified this effect by injecting fake sources into  their observations to determine the recovery efficiency and loss of detection depth from analysis cuts. Here, we forego  such analysis to focus the discussion on the effects of contamination in  optical follow-up of GW events.}. We calculate $\sigma$ as the  quadrature sum of the flux errors from the first and third epochs  ($\sigma = \sqrt{\sigma_1^2 + \sigma_3^2}$, where $\sigma_1$ and $\sigma_3$ are the flux errors from the first and third epochs, respectively). This criterion leaves a sample of 48 objects.

\item We reject sources that exhibit a significant ($\ge 3\sigma$) rise in flux  between the first and second epochs or the second and third epochs to eliminates variable  sources that do not decline steadily. This criterion leaves a sample of 32 objects.

\item The remaining 32 candidates from step 4 undergo visual inspection. We reject sources that are present as a point source in the fourth (template) epoch that do not have a galaxy within 20\arcsec. Sources are cross-checked against NED\footnote{\singlespace \url{https://ned.ipac.caltech.edu/}}  and SIMBAD\footnote{\singlespace \url{http://simbad.u-strasbg.fr/simbad/}}. This criterion is designed to remove variable stars and long-timescale transients.
\end{enumerate}

\clearpage
Only four events passed our final criterion. We find that two of those events are  coincident with the nuclei of known AGN (PKS\,0129-066 and Mrk\,584), indicating  that they represent AGN variability. A third candidate is coincident with the nucleus of the bright radio source PMN\,J0203+0956 ($F_\nu(365\,{\rm MHz})\approx 0.4$ Jy,  \citealt{Douglas+96}), also suggesting AGN variability.

The final candidate in our search is located at RA = 01$^{\rm h}$42$^{\rm m}$16$^{\rm s}$.17  and DEC = $-$02\arcdeg13\arcmin42 6\arcsec~(J2000), with an offset of $5.8$ arcsec from the galaxy  CGCG 386-030 (RA =  01$^{\rm h}$42$^{\rm m}$15$^{\rm s}$.6, DEC = $-$02\arcdeg13\arcmin38  5\arcsec; J2000), at $z = 0.041$ or $d_L \approx 187$ Mpc (6dFGS, \citealt{Jones+04,Jones+09}); see  \Cref{fig:ch4_PS15cdi}. We note that this distance is inconsistent with the 90\% confidence interval for the distance to GW151226 based on the GW data \citep{LIGOGW151226}. We observe this source in a state of rapid decline with an  absolute magnitude of $M_i \approx -15$~mag on 2015 December 28 and $M_i\approx -14.5$~mag on 2016  January 1, indicating a decline rate of $\approx 0.12$ mag d$^{-1}$; the decline rate in $z$-band is  $\approx 0.10$ mag d$^{-1}$. Additionally, the source exhibits a red $i-z$ color of $0.3$ mag.

We fit these data to a power-law model typical for GRB afterglows $(F_\nu \propto \nu^{\beta} t^{\alpha})$  and find a temporal index of $\alpha = -0.43\pm0.12$ and a spectral index of $\beta = -1.8\pm0.8$,  both of which differ from the expected values for GRB afterglows ($\alpha \approx -1$,  $\beta \approx -0.75$, \citealt{Sari+98}). Additionally, we compare our observations to a  kilonova model with ejecta parameters of $v_{\rm ej} = 0.2$c and $M_{\rm ej} = 0.1$ M$_\odot$ \citep{BarnesKasen13}. We find that the timescale of the transient agrees with those expected for kilonovae,  but the color is bluer than the expected value of $i-z \approx 1$~mag \citep{BarnesKasen13}. Thus, the properties of this transient differ from those of GRB afterglows or kilonovae. The observations  and models are shown in \Cref{fig:ch4_PS15cdi}.

\begin{figure}[t!]
\centering
\includegraphics[width=0.9\textwidth]{./figs/chapter4/fig2.pdf}
\caption{\singlespace {\it Top}: Single-epoch images of our main candidate from all four epochs (green circle). This is the event discovered as PS15cdi in the PSST about 94 d prior to GW151226. {\it Bottom}: Light curve data for PS15cdi from PSST $w$- and $i$-band observations (green squares and yellow diamonds, respectively). Our DECam $i$- and $z$-band data are shown as blue circles and red stars, respectively. The revised DECam analysis  using pre-existing templates is shown as open symbols. Upper limits are indicated by triangles. The inset focuses on our DECam data, indicating a rapid decline in both $i$ and $z$ bands.  We fit a power-law model to the data finding a temporal index of  $\alpha = -0.43$ (dashed-dot line). Kilonova models from \citet{BarnesKasen13} with $v_{\rm ej}=0.2c$ and $M_{\rm ej}=0.1$ M$_\odot$  at a distance of $187$~Mpc are also shown (dashed line).}
\label{fig:ch4_PS15cdi}
\end{figure}

This source was previously detected as PS15cdi on 2015 September 23 by the Pan-STARRS  Survey for Transients (PSST\footnote{\singlespace \url{http://psweb.mp.qub.ac.uk/ps1threepi/psdb/candidate/1014216170021342600/}}, \citealt{Huber+15}); see \Cref{fig:ch4_PS15cdi}. The absolute $i$-band  magnitude in the first PSST epoch, $M_i\approx -16.6$ mag and the shallow decline of $\approx0.6$~mag  over $\approx70$~d, are consistent with a Type IIP core-collapse supernova (SN).  A likely interpretation of the rapid decline in our observations is that PS15cdi is a Type IIP SN undergoing  the rapid transition from the hydrogen recombination driven plateau to the radioactive $^{56}$Co dominated  phase \citep{KasenWoosley09,Sanders+15,Dhungana+16}. The red $i-z$ color in our data is consistent with observations  of other IIP SN during this phase of evolution (e.g., SN2013ej, \citealt{Dhungana+16}). This transition typically occurs about 100~d post explosion \citep{KasenWoosley09,Sanders+15,Dhungana+16}, consistent with the  timing of our observations relative to the first detection in PSST.

To mitigate the effect of excess flux from PS15cdi still present in our template observations, we repeat the analysis  using as templates archival DES $i$- and $z$-band images from 2013 December 19. These data were processed  and image subtraction was performed as described in \cref{sec:ch4_obs}. We find that flux from PS15cdi is indeed still present  in our original template image, leading to revised first epoch absolute magnitudes of $M_i \approx -15.6$ and  $M_z \approx -16$~mag, and a decline rate between the first and fourth epochs of 0.04~mag d$^{-1}$, in both  $i$- and $z$-bands. The transient still exhibits a red $i-z$ colors of $\approx 0.4$~mag across all four epochs.

\clearpage
Clearly, we can rule out this candidate based on the PSST detections prior to  GW151226, but without this crucial information this candidate would have been a credible optical counterpart based on its light curve behavior and distance.  It is therefore useful to develop an understanding of the expected rates for such contaminants to inform expectations in future searches. We adopt a local core-collapse SN rate of   $7\times10^{-5}$~yr$^{-1}$~Mpc$^{-3}$ \citep{Li+11,Cappellaro+15},  and a Type IIP SN fraction of 48\% of this rate \citep{Smith+11}. The rapid decline phase typically lasts about $20$~d \citep{KasenWoosley09,Sanders+15,Dhungana+16},  so we consider events that occur within that time frame. Lastly, given its apparent brightness,  we assume that PS15cdi represents the approximate maximum distance to which we can observe these events in our data. We thus find an expected occurrence rate of $\sim 0.04$ events in our search area making our detection of PS15cdi somewhat unlikely, and indicating that $\lesssim 1$ such  events are expected in a typical GW localization region.

Our detection of PS15cdi clearly demonstrates the presence and impact of contaminants when conducting optical follow-up of GW events. Core-collapse SNe are generally not considered to be a significant contaminant due to their much longer timescales compared to kilonovae (e.g., \citealt{CowpBerger15}). However, a source like PS15cdi, caught in a rapid phase of its evolution  despite its overall long timescale, and exhibiting a relatively red color could satisfy a set of criteria designed for finding kilonovae ($\Delta m\gtrsim 0.1$ mag d$^{-1}$ and $i-z\gtrsim 0.3$ mag; \citealt{CowpBerger15}).

\clearpage
The most effective approach to deal with contaminants like PS15cdi is rapid, real-time identification. Once a candidate is deemed interesting, optical spectroscopy and  NIR photometry can quickly distinguish between a SN or kilonova/afterglow origin. Specifically, the kilonova spectrum will be redder, with clear suppression below $\sim 6000$~\AA~due to the opacities of $r$-process elements \citep{Kasen+13}. By comparison, the SN spectrum will appear bluer and dominated by iron group opacities \citep{Kasen+13}, while the afterglow spectrum will exhibit a featureless power-law spectrum \citep{Berger2014}. If pre-existing templates are not available then the significant aspect is rapid initiation of follow-up observations at $\lesssim 1$ d that can distinguish the rising phase of a kilonova or  off-axis GRB from a declining SN.

\section{Conclusions}
\label{sec:ch4_conc}
We presented the results of our deep optical follow-up of  GW151226 using the DECam wide-field imager. Our observations cover a sky area of 28.8 deg$^2$, corresponding to $3\%$ of the initial {\tt BAYESTAR} probability map and 2\% of the final {\tt LALInference} map. We obtained four epochs of observations starting 10 hours after the  event was announced and spanning 2--24 days post trigger, with an average $5\sigma$ point-source sensitivity of  $i\approx21.7$ and $z\approx21.5$, with an epoch-to-epoch scatter of 0.4 mag, in our difference images.

Using the final epoch as a template image, we searched for sources that display a significant and steady decline in brightness throughout our observations, and which are not present in the template epoch. This search yielded four transients, of which three result from AGN variability. The final event is located at a distance of about 187 Mpc offset by $5.8''$ from its host galaxy. It also broadly possesses the observational features of a kilonova in terms of its rapid decline and red $i-z$ color. However, this source corresponds to the transient PS15cdi, which was discovered in PSST about 94 days prior to the GW trigger. It is a likely Type IIP supernova, which our observations caught in the steep transition at the end of the plateau phase. The detection of this event indicates that careful rejection of contaminants, preferably in real time, is essential in order to avoid mis-identifications of optical counterparts to GW sources.
