%No preamble material
The direct detection of gravitational waves from distant astrophysical sources has opened up and new and exciting view of the cosmos, not unlike the first detections of celestial objects in radio or x-rays. Gravitational wave observations allow new probes of Einstein's General Theory of Relativity in regimes not previously accessible and illuminate the existence of astrophysical objects that were previously invisible to purely electromagnetic observations. The true power of gravitational wave detections become apparent when they are paired with information from traditional electromagnetic telescopes. This emerging field of joint gravitational wave-electromagnetic (GW-EM) astronomy presents an unparalleled opportunity to test our understanding of the Universe.

This thesis presents a series of studies, both observational and theoretical, into developing and executing observational programs that facilitate the detection of an electromagnetic counterpart coincident with a gravitational wave detection. These studies first focus on detailed investigations into the design of optical follow-up programs that overcome many of the hurdles expected to arise when searching for these elusive electromagnetic counterparts. More recently, this work has focused on actually conducting follow-up searches, including the first joint detection of gravitational waves and electromagnetic waves.

In this chapter, we present an introduction to the growing field of joint GW-EM astronomy. This includes a brief introduction to gravitational waves, their sources in the Universe, and how we detect them (\cref{sec:intro_gw}), an overview of the expected electromagnetic counterparts and how we conduct searches for them (\cref{sec:intro_joint}), and lastly a history of gravitational wave detections leading up to the first joint GW-EM discovery (\cref{sec:intro_history}).

\section{Gravitational Waves}
\label{sec:intro_gw}
The existence of gravitational waves (GW) was first investigated by Einstein shortly after he introduced the world to his general theory of relativity \citep{Einstein1916}. Gravitational waves are ripples in both space and time that propagate through the universe at the speed of light, carrying with them energy, momentum, and most importantly information about their progenitors. Mathematically, gravitational waves are most simply thought of as a small perturbation to a flat metric \citep[$g_{\mu\nu}$; ][]{Misner+73,Carroll2004}:
\begin{equation}
\label{eq:intro_metric}
g_{\mu\nu} = \eta_{\mu\nu} + h_{\mu\nu}, \quad ||h|| \ll 1
\end{equation}
\noindent where $\eta_{\mu\nu}$ is the flat Minkowski spacetime given by $\eta_{\mu\nu} = {\rm diag}(-1, 1,1,1)$ and $h_{\mu\nu}$ is called the spacetime {\em strain} and can be loosely thought of as the amplitude of the gravitational wave. This linearized approach leads to a solution representing gravitational plane waves \citep[see e.g.,][and references therein]{Misner+73,Carroll2004}.

While discussions of this plane wave solution and the detailed nature of gravitational waves are beyond the scope of this thesis, there are several important physical aspects of gravitational waves worth mentioning. Linear/Angular momentum -> quadrupole emission (equation). 

\subsection{GW Sources}
\label{sec:intro_gw_source}
EM Radiation is accelerating charges. GW is accelerating mass. Small strain scales mean that we have to consider drastic accelerations of a lot of mass

Relevant to this thesis is compact object binary mergers. 1-2 paragraphs

\subsection{GW Detectors}
\label{sec:intro_gw_det}
GW radiation is nice because it does not disperse but makes them hard to detect

A brief paragraph about the early bar measurements, HT Pulsar, focus on context of thesis

1-2 paragraphs on ground-based detectors, functionality, and sensitivity 

\section{Joint GW-EM Astronomy}
\label{sec:intro_joint}
Talk about why we want to do joint GW-EM astronomy. Can do a lot of cool physics with GW, but Gw-EM is better. What are the science gains and limitations of individual detections vs. Joint

\subsection{EM Counterparts}
\label{sec:intro_counterparts}
Important to understand the range of possible emission

GRB

On-Axis Jet

Off-Axis Jet

Kilonova (1-2 paragraphs)

Precursors (Neutron/Cocoon)

Late-time radio emission.

\subsection{Search Methods}
\label{sec:intro_searchmethods}
Once you know what you are looking for, need to understand how to find it. 

Major difficulty is large localization regions 

\subsubsection{Galaxy-Targeted}
\label{sec:intro_galaxy}
General idea of galaxy targeted searches

Pros - Available to telescopes with small FoV, in small localization regions potentially quicker than wide-field.

Cons - Localization and distance errors matter. Can quickly encounter more galaxies than doable per night

\subsubsection{Wide-Field Searches}
\label{sec:intro_widefield}
General idea of wide field searches and instruments involved

Pros - Cover whole region quickly with little bias, place limits on non-detections

Cons - Contamination and time available

\section{Detections with Advanced LIGO/Virgo}
\label{sec:intro_history}
Advanced LIGO time line and lead up to first observing run 

\subsection{GW150914: The First Detection}
\label{sec:intro_gw150914}
The GW detection of GW150914. First BBH. Surprisingly massive.

Massive EM follow-up campaign.

The \fermi\ stuff

Ushered in exciting new era

\subsection{Binary Black Holes. So Many.}
LIGO would have continued success during O1 detecting a second BBH merger and strong candidate

GW170104 - The most distance detection

GW170814 - Virgo joining LIGO, the first three detector BBH merger, massive effect on localization 

\subsection{GW170817: The Dawn of Joint GW-EM Astronomy}
The detection of GW170817. GW information and localization due to VIRGO

GRB and multi-wavelength detections

Optical counterpart and observations. 


%Don't include any calls to \bibliography
