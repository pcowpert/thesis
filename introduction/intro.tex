%No preamble material
The direct detection of gravitational waves from distant astrophysical sources has opened up a new and exciting view of the cosmos, not unlike the first detections of celestial objects in radio or x-rays. Gravitational wave observations allow new probes of Einstein's General Theory of Relativity in regimes not previously accessible and illuminate the existence of astrophysical objects that were previously invisible to purely electromagnetic observations. The true power of gravitational wave detections become apparent when they are paired with information from traditional electromagnetic telescopes. This emerging field of joint gravitational wave-electromagnetic (GW-EM) astronomy presents an unparalleled opportunity to test our understanding of the Universe.

This thesis presents a series of studies, both observational and theoretical, into developing and executing observational programs that facilitate the detection of an electromagnetic counterpart coincident with a gravitational wave detection. These studies initially focus on detailed investigations into the design of optical follow-up programs that overcome many of the hurdles expected to arise when searching for elusive electromagnetic counterparts. More recently, this work has focused on actually conducting follow-up searches, including the first joint detection of gravitational and electromagnetic waves.

In this chapter, I present an introduction to the growing field of joint GW-EM astronomy. This includes a brief introduction to gravitational waves, their sources in the Universe and how they are detected (\cref{sec:intro_gw}), an overview of the expected electromagnetic counterparts and how we conduct searches for them (\cref{sec:intro_joint}), and lastly a history of gravitational wave detections leading up to the first joint GW-EM discovery (\cref{sec:intro_history}).

\section{Gravitational Waves}
\label{sec:intro_gw}
\subsection{Gravitational Wave Basics}
\label{sec:intro_gw_basics}
The existence of gravitational waves (GW) was first investigated by Einstein shortly after he introduced the world to his general theory of relativity \citep[GR;][]{Einstein1915,Einstein1916,Einstein1918}. Gravitational waves are ripples in both space and time that propagate through the Universe at the speed of light; carrying with them energy, momentum, and most importantly information about their progenitors. Mathematically, gravitational waves are most simply thought of as a small perturbation to a flat metric \citep[$g_{\mu\nu}$; ][]{Misner+73,Carroll2004}:
\begin{equation}
\label{eq:intro_metric}
g_{\mu\nu} = \eta_{\mu\nu} + h_{\mu\nu}, \quad ||h|| \ll 1
\end{equation}
\noindent where $\eta_{\mu\nu}$ is the flat Minkowski spacetime given by $\eta_{\mu\nu} = {\rm diag}(-1,1,1,1)$ and $h_{\mu\nu}$ is called the spacetime {\em strain} which can be loosely thought of as the amplitude of the gravitational wave. This linearized approach leads to a solution representing gravitational plane waves \citep[see e.g.,][and references therein]{Misner+73,Carroll2004}.

While discussions of this plane wave solution and the detailed nature of gravitational wave radiation are beyond the scope of this thesis, there are several important physical aspects worth mentioning. First, in contrast to electromagnetic waves which radiate as a dipole, conservation of linear and angular momentum require that gravitational waves radiate in a quadrupole fashion \citep[see e.g.,][and references therein]{Centrella+10}. The nature of this quadrupolar emission is often expressed using two orthogonal polarizations: $h_{+}$ which represents linear polarization and $h_{\times}$ which is oriented 45\degr~relative to $h_{+}$. This also suggests that sources which produce gravitational waves must have an accelerating quadrupole moment, analogous to an accelerating dipole moment for electromagnetic radiation. The resulting strain from such an acceleration is given by \citep{Centrella+10}:
\begin{equation}
\label{eq:intro_quad}
h \apx \frac{G}{c^4} \frac{\ddot{Q}}{r}
\end{equation}
\noindent where $Q$ is the quadrupole moment of the source and $r$ is the distance to the source. \Cref{eq:intro_quad} also points to our second important point about the behavior of gravitational waves which is that the strain decreases as~$h \apx r^{-1}$. These two physical properties have important implications for astrophysical sources of interest (\cref{sec:intro_gw_source}) and the design of gravitational wave detectors (\cref{sec:intro_gw_det}).

\subsection{Gravitational Wave Sources}
\label{sec:intro_gw_source}
Much like EM radiation arises from the acceleration of charged particles, gravitational radiation is the result of rapidly accelerating masses. This can be seen by rewriting \Cref{eq:intro_quad} as \citep{Centrella+10}:
\begin{equation}
\label{eq:intro_quad}
h \apx \frac{G}{r\,c^2} (M_{\rm quad} \beta^2)
\end{equation}
\noindent where $M_{\rm quad}$ is the accelerating mass and $\beta \equiv v / c$ is the velocity of the source. Given that the factor of $G / c^2$ is on the order of $\apx7\times10^{-29}$~cm~g$^{-1}$, this indicates that the production of detectable gravitational wave sources requires large masses moving at relativistic speeds.

There are many potential sources of gravitational waves in the Universe \citep[see e.g,][]{CutlerThorne02} including: pulsars \citep{Hewish+68, Jaranowski+98}, asymmetries in core collapse supernovae \citep[SN;][]{Dimmelmeier+01,Ott+06,Ott2009}, or the effects of inflation in the very early universe \citep{Turner1997,Maggiore2000}. In the context of this thesis, the most interesting astrophysical source of gravitational waves is the inspiral and merger of compact object binaries \citep[see e.g.,][]{Belczynski+02,Blanchet2002,Blanchet2006}.

Compact object binaries are binary systems comprised of some combination of neutron stars (NS) and/or black holes (BH). As these massive objects orbit around each other they experience energy loss due to gravitational wave emission. Indirect evidence of this gravitational wave emission can be seen in radio observations of binary pulsar systems \citep{HulseTaylor75,Weisberg+05}. As the binary system loses energy, the binary separation decreases until eventually the binary components coalesce and merge. This final merger produces a large ``chirp" of gravitational waves followed by a ``ringdown" as the merger remnant stabilizes \citep{Centrella+10}. During the final seconds of this inspiral and merger, the system radiates gravitational waves with a luminosity of \citep{FaberRasio12}:
\begin{equation}
\label{eq:intro_gwlum}
L_{\rm GW} \approx 8.7 \times 10^{51} q^2 (1+q) \left ( \frac{M_1}{1.4~\msun} \right )^5 \left ( \frac{a}{100~{\rm km}} \right )^{-5}
\end{equation}
\noindent where $M_1$ is the mass of the primary, $q$ is the binary mass ratio $q \equiv M_2 / M_1$, and $a$ is the binary separation. Thus, for an equal mass ratio Binary Neutron Star (BNS; $M_1 = M_2 = 1.4~\msun$, $q = 1$) system separated by just 30~km (e.g., a few NS radii), the gravitational wave luminosity is an immense $L_{\rm GW} \apx 10^{54}$~erg~s$^{-1}$. A more relevant physical quantity for the detection of the gravitational waves here on Earth is the amplitude of the strain, $h$, given by \citep{FaberRasio12}:
\begin{equation}
\label{eq:intro_gwh}
h \approx 5.5 \times 10^{-23} q \left ( \frac{M_1}{1.4~\msun} \right )^2 \left ( \frac{a}{100~{\rm km}} \right )^{-1} \left ( \frac{r}{100~{\rm Mpc}} \right )^{-1}
\end{equation}
\noindent at a frequency of \citep{FaberRasio12}:
\begin{equation}
\label{eq:intro_gwf}
f_{\rm GW} \approx 195 \left ( \frac{M}{2.8~\msun} \right )^{1/2} \left ( \frac{a}{100~{\rm km}} \right )^{-3/2}~{\rm Hz}
\end{equation}
\noindent where $r$ is the distance to the source. In the case of our BNS example above, at a distance of 100~Mpc, the expected strain amplitude is $h \apx 10^{-22}$ at a frequency of $f_{\rm GW} \apx 1200$~Hz. Therefore, while gravitational waves from compact binaries can be some of the most luminous events in the Universe, the actual scale of the spacetime distortion is almost imperceptible.

\subsection{Gravitational Wave Detectors}
\label{sec:intro_gw_det}
The small strain amplitude calculated in \cref{sec:intro_gw_source} hints at the unique challenge involved in making a direct detection of gravitational waves. While gravitational waves can propagate for vast distances through space with little dissipation, their very weak interaction with matter means that any detector must achieve previously unprecedented levels of sensitivity. The current best instruments for the detection of gravitational waves from the inspiral and merger of a compact object binary are large ground-based interferometers. The existing network of such detectors consists of the two Advanced Laser Interferometer Gravitational-Wave Observatory (LIGO) detectors with sites in Hanford, Washington (H1) and Livingston Parish, Louisianna \citep[L1;][]{LIGOEarlyRef,LIGOMainRef}, the Virgo Interferometer in Italy \citep{VirgoC7Burst,VIRGORef}, and the GEO600 interferometer in Germany \citep{GEOMainRef}. Future additions to this network include the KAGRA interferometer in Japan \citep{KAGRA}, and a new LIGO installation in India \citep[LIGO-India;][]{LIGOIndia}.

All of these detectors are, at their heart, a specialized Michelson Interferometer. They operate by passing a highly stabilized laser through a beam splitter, causing the beam to pass down long ($4$~km for LIGO) orthogonal arms, bounce off mirrors functioning as test masses, and return to a photodiode. The system is tuned such that the lasers destructively interfere at the photodiode, producing no signal \citep[see e.g.,][and references therein]{LIGOMainRef,Adhikari2014}. As a gravitational wave passes through the detector, the lengths of the arms change causing a signal to appear on the photodiode. The detectors are consequently sensitive to a fractional change in length that is roughly proportional to the strain amplitude (e.g., $h \apx \Delta L / L$). For a sense of scale, the Advanced LIGO interferometer at design sensitivity will be able to detect strain amplitudes of $h \apx 10^{-22}$, or a change of length in the 4 km arms of just $\Delta L \apx 10^{-19}$ meters. This sensitivity is sufficient to detect the merger of a BNS at an average distance of $\apx 200$~Mpc or the merger of a more massive binary black hole (BBH) system out to a gigaparsec.

Despite the phenomenal sensitivity of any single detector, it is necessary that they operate in a collective network for several key reasons. The first, and most important reason, is coincident detection of signals. The extremely sensitive nature of these interferometers makes them highly susceptible to noise that can produce a false GW signal. Therefore, GW signals appearing in multiple instruments, separated by the light travel time between interferometer locations, provides a powerful constraint on the physical nature of detections. The second reason is that the wide antenna pattern of a single interferometer makes it difficult to determine the precise sky location of any GW source. In a network of two detectors, degenerate triangulation based on the GW arrival times combined with information about the GW polarization at each detector can produce sky localizations that are $\apx1000$ deg$^2$ in size \citep[see e.g.,][]{LIGOLocalization,ChenHolz16}. If a third detector is added, then the localization can be improved by an order of magnitude to just $\apx100$~deg$^2$, finally improving to just $\apx10$~deg$^2$ in the final five detector network \citep[see e.g.,][]{LIGOLocalization,ChenHolz16}.

\section{The Primary Considerations for Joint GW-EM Astronomy}
\label{sec:intro_joint}
The ability to obtain routine direct observations of gravitational waves with instruments such as Advanced LIGO and Virgo unlocks entirely new avenues to study compact object binaries and to probe our understand of fundamental physics. However, in order to maximize the science returns from such detections it is necessary to identify a coincident electromagnetic counterpart. The identification of an EM counterpart provides numerous benefits including: improved localization leading to host galaxy identification, determination of the distance and energy scales, identification of the progenitor local environment, and insight into the hydrodynamics of the merger \citep[see e.g.,][]{Sylvestre2003,Stubbs2008,Phinney2010,Stamatikos+09, Fong+10,MetzgerBerger12,Fong+13,FongBerger13,Fong+15}. Furthermore, identification of the EM counterpart facilitates other fields of study such as determining the primary sites of heavy $r$-process element production \citep{Rosswog+14,vandeVoort+15,Kasen+17}, placing limits on the NS equation of state \citep[see e.g.,][]{Hotokezaka+11,Kawaguchi+15,Radice+18}, and making independent measurements of the local Hubble Constant \citep[H$_0$; see e.g.,][]{Schutz1986,HolzHughes05,DelPozzo2012, LIGOH0,Guidorzi+17}.

\subsection{The Zoo of EM Counterparts}
\label{sec:intro_counterparts}
The first step to conducting successful follow-up searches for electromagnetic signals associated with a GW event is understanding the range of potential counterparts. In the case of two black holes merging, there has been extensive theoretical research attempting to predict possible electromagnetic counterparts \citep[see e.g.,][]{Krolik2010,Loeb2016,Perna+16,DOrazioLoeb17,Kelly+17}. However, these studies have been unable to identify a clear ubiquitous counterpart that will be detectable on Earth. The more promising case is the merger of a compact object binary containing at least one neutron star. The presence of matter supplied by the NS opens up a plethora of potential electromagnetic counterparts \citep[see e.g.,][]{MetzgerBerger12,Piran+13}.

The most energetic counterpart is a short-hard gamma-ray burst \citep[SGRB; e.g.,][]{Paczynski1986,Narayan+92,Berger2014} powered by rapid accretion onto the compact remnant which leads to a strongly beamed relativistic jet. The evidence for a connection between SGRBs and compact object mergers includes the association with older stellar populations than for long GRBs \citep{Berger+05,Bloom+06,Berger2011,Fong+11,Fong+13,Fong+15}, large physical offsets between the burst locations and their host galaxies indicative of progenitor kicks \citep{Fong+10, FongBerger13,Fong+15}, weak association with the underlying stellar light of their hosts \citep{Fong+10, FongBerger13,Fong+15}, and a lack of association with supernovae \citep{Fox+05,Soderberg+06,Berger2010,FongBerger13,Berger2014,Fong+15}. The relativistic jet can also be observed in the x-ray, optical, and radio as non-thermal afterglow emission arises from interaction with the ambient medium. This emission is also affected by relativistic beaming but may still be seen off-axis as the jet decelerates and expands. In particular, for large off-axis angles $(\theta\gtrsim 2 \theta_j)$ the emission will be most pronounced in the radio on timescales of weeks to years  \citep{NakarPiran11,MetzgerBerger12}.

These mergers are also expected to produce ejecta through dynamical processes such as tidal forces and accretion disk winds \citep{Goriely+11,Bauswein+13a,Fernandez+15,Radice+16,Metzger2017,SiegelMetzger17}. Numerical simulations indicate that the unbound debris has a mass of $M_{\rm ej} \sim 10^{-4} - 10^{-1}$~M$_{\odot}$ with a velocity of $\beta_{\rm ej} \sim 0.1 - 0.3$, with a dependence on parameters such as the mass ratio and equation of state \citep{Hotokezaka+13,Tanaka+14,Kyutoku+15}. The ejecta are expected to be neutron rich, with a typical electron fraction of $Y_e \lesssim 0.3$ with simulations showing a range of values from $Y_e \apx 0.1 - 0.3$ \citep{Goriely+11,Bauswein+13a,Sekiguchi+15, Radice+16,SiegelMetzger17}. This electron fraction is low enough ($Y_e \lesssim 0.25$) that the ejecta are expected to undergo $r$-process nucleosynthesis, producing heavy elements $(A \gtrsim 130)$, particularly in the lanthanide and actinide groups \citep{Goriely+11,Bauswein+13a,SiegelMetzger17}. These groups of elements have open f-shells which allow a large number of possible electron configurations, resulting in a large opacity in the bluer optical bands ($\kappa_{\nu} \sim$ 100 cm$^2$ g$^{-1}$ for $\lambda \sim 1$~$\mu$m, see e.g. Figure 10 in, \citealt{Kasen+13}). A more recent calculation by \cite{Fontes+17} suggests that the lanthanide opacities may be an order of magnitude higher ($\kappa_{\nu} \sim 1000$  cm$^2$ g$^{-1}$ for $\lambda \sim 1$~$\mu$m).

Radioactive decay of the $r$-process elements synthesized during the merger heats the ejecta producing an isotropic, thermal transient \citep{LP98,Rosswog2005, Metzger+10,Tanaka+14,Metzger2017}. The combination of low ejecta mass and high ejecta velocity, coupled with the strong optical line blanketing, results in a transient that is faint ($i \approx 23$ and $z \approx 22$ mag at 200 Mpc), red ($i-z \gtrsim 0.3$ mag), and short-lived with a typical duration of $\sim {\rm few}$ days in $z$-band and $\sim{\rm week}$ in $J$-band \citep{BarnesKasen13,Barnes+16,Metzger2017}. In the case of larger opacities \citep[e.g.,][]{Fontes+17}, the transient is expected to peak in the IR (\apx3~$\mu$m), with a duration of \apx10~days \citep{Fontes+17,Wollaeger+17}.

In addition to the neutron-rich dynamical ejecta, recent work has suggested that these mergers can also produce ejecta with a high electron fraction $(Y_e > 0.25$,  \citealt{Wanajo+14,Goriely+15}) if the merger remnant is a hypermassive neutron star \citep[HMNS; see e.g.,][]{Sekiguchi+11} with a lifetime of $\gtrsim 100$ ms. The resulting HMNS irradiates the disk wind ejecta with a high neutrino luminosity which raises the electron fraction of the material and suppresses $r$-process nucleosynthesis. This material will have an opacity similar to the Fe-peak opacities seen in Type Ia SNe, producing emission that is slightly brighter ($r \approx 22$ mag at 200 Mpc), bluer ($i-z \lesssim 0$), and shorter-lived ($\approx 1-2$ days, \citealt{MetzgerFernandez14,Kasen+15}). This blue kilonova component has a strong dependence on viewing angle, with the polar regions of the ejecta being exposed to the highest neutrino flux \citep{MetzgerFernandez14,Kasen+15}. Consequently, if the merger is viewed face-on $(\theta \approx$ 15--30 deg), this blue component may be visible. We expect that up to half of mergers will be viewed at such angles \citep[see e.g.,][]{MetzgerBerger12}. While this component will be brighter than the expected $r$-band emission from the lanthanide-rich material, its detection requires rapid-cadence observations within a few hours of the GW detection \citep{CowpBerger15}. However, at larger viewing angles, the lanthanide-rich material in the dynamical ejecta will obscure the blue emission, and only a red kilonova will be observed \citep{Kasen+15,Metzger2017}.

Lastly, it has been argued that a small fraction of the merger ejecta may expand so rapidly that it is unable to undergo $r$-process nucleosynthesis \citep{Bauswein+13a}. This material instead deposits energy into the ejecta via neutron $\beta$-decay. At very early times $(\lesssim 1$ hr post merger), the specific heating rate from the neutron $\beta$-decay is an order of magnitude higher than that generated by the $r$-process nuclei \citep{Metzger2017}. This timescale is well matched to the diffusion time for the free neutron ejecta resulting in bright brief emission. For an ejecta mass with $10^{-4}$ M$_{\odot}$ of free neutrons, the resulting transient will have a peak $r$-band magnitude of $\approx 22$ mag at 200 Mpc with a characteristic timescale of $\sim 1-2$ hours \citep{Metzger+15}. This speculative early time emission is often referred to as a ``neutron precursor." However, due to the high velocity of the free neutrons $>0.4$c, this component of the ejecta may be visible before the equatorial lanthanide-rich ejecta and thus be observable for a wider range of viewing angles than the blue kilonova. This transient will be as bright as the blue kilonova emission but significantly shorter in duration requiring particularly rapid observations in response to a GW trigger \citep{CowpBerger15}.

To summarize, kilonova emission with red colors $(i-z\gtrsim 0.3)$, a peak brightness of $z\approx22.2$~mag at 200 Mpc, and a duration of $\sim \rm few$ days is expected to be ubiquitous. Blue kilonova emission due to a surviving HMNS is expected to be bluer ($i-z \lesssim 0$~mag), similarly faint ($r \approx 22$~mag) and shorter in duration ($\lesssim 1-2$ days). The observed behavior and composition of the dynamical ejecta and associated red kilonova emission is expected to be robust and ubiquitous, however the nature and observability of the blue kilonova emission depends both on the fraction of cases in which a HMNS survives (unknown) and on geometrical effects ($\lesssim 50\%$).  Finally, emission due to free neutrons will be similar in brightness and color to the blue kilonova emission, but with a timescale of only $\sim \rm few$ hours; the prevalence of this signal is uncertain.

\subsection{Considerations for Search Methods}
\label{sec:intro_searchmethods}
Once a population of suitable counterparts has been identified (\cref{sec:intro_counterparts}) the next step is designing strategies for follow-up and searches. While the energetic nature of SGRBs combined with with a rich history of follow-up of SGRB afterglows \citep[see e.g.,][for a review]{Berger2014} suggests that searches using gamma-ray facilities is ideal, in actuality the beamed nature of the relativistic jet means that this emission may not be observable for all mergers detected by LIGO \citep{ChenHolz13}. The most promising counterpart of those outlined in \cref{sec:intro_counterparts} is instead the optical/NIR kilonova emission \citep{MetzgerBerger12}. The ubiquitous and isotropic nature of the kilonova, combined with the vast array of optical telescopes available across all aperture scales, makes follow-up searches in the optical regime the most compelling approach. There are two primary approaches to these searches: ``galaxy-targeted" searches and ``wide-field" searches.

\subsubsection{Galaxy-Targeted Searches}
\label{sec:intro_galaxy}
The primary motivation behind galaxy-targeted searches is that by only observing the potential host galaxies of the merger, one can drastically reduce the amount of time it takes to search the large GW localization region (see \cref{sec:intro_gw_det}). This strategy involves convolving a GW localization region with an appropriately constructed host galaxy catalog \citep[e.g., the Gravitational Wave Galaxy Catalog, GWGV;][]{GWGCRef} to produce an optimized list of galaxies to observe \citep{Gehrels+16,Singer+16a,Singer+16b}. In addition to potentially reducing the amount of time spent observing, this has the added bonus of allowing small field-of-view (FoV) telescopes to participate in searches. For example, this is the dominant technique for follow-up by the \swift\ mission \citep{Gehrels+04,Kanner+12}. Lastly, galaxy-targeted searches are less likely to be affected by contamination due to false-positives from unrelated transients \citep{Singer+16a}.

The galaxy-targeted method does have several downsides. The first is that it relies on the existence of complete and unbiased galaxy catalogs to guide observations, meaning that potential host galaxies can be missed. Second, assuming that kilonovae and SGRBs share similar offsets from their hosts \citep[see e.g.,][]{FongBerger13}, then certain small FoV instruments may not observe a sufficient region surrounding the galaxy to identify the counterpart. Lastly, as the sensitivity of instruments like LIGO continues to improve, the increasing number of galaxies per localization volume means that galaxy-targeted searches will become unfeasible.

\subsubsection{Wide-Field Searches}
\label{sec:intro_widefield}
In contrast to galaxy-targeted searches, wide-field searches rely on being able to observe the entire GW localization region by efficiently tiling the region using wide FoV instruments. Several current instruments running wide-field transient surveys are well suited for this task including: the Dark Energy Camera \citep[DECam;][]{Flaugher+15}, the Pan-STARRS Survey \citep{Chambers+16}, and the Zwicky Transient Facility \citep[ZTF;][]{ZTFRef1,ZTFRef2}. These wide-field searches have the advantage of being able to observe the entire region without the observational biases present in the galaxy-targeted searches. Furthermore, by covering the entire region these studies are able to place statistical upper limits on the EM emission from compact binary mergers, providing scientific insight even in the case of non-detections.

The primary downside of wide-field searches is that they must contend with the challenge of ruling out false positives that arise from unrelated contaminating transients \citep{CowpBerger15}. Observing even $\apx100$~deg$^2$ of sky over the course of a night will reveal numerous new transients of both galactic and extragalactic origin. While the rates of these contaminants are expected to be low \citep[$\mathcal{R} \apx 2$ events per deg$^2$][]{Cowp+17}, they still present a challenge when trying to identify the presence of a genuine EM counterpart in the data. The other major challenge facing wide-field searches is simply observing time available \citep{CowpBerger15}. It may not be possible to observe an entire localization in a single night of observing leading to uneven cadence and coverage which will weaken the analysis.

\section{The History of Detections with Advanced LIGO/Virgo}
\label{sec:intro_history}
I conclude this introduction with a discussion of recent advancements in the field of joint GW-EM astronomy and how they relate to this thesis. This includes the planning of follow-up prior to the first detection, follow-up efforts for the first and subsequent detections, and ultimately the first joint detection of gravitational waves and electromagnetic radiation. The work presented in this thesis uniquely traces the development for each of these steps.

The Advanced LIGO interferometer was scheduled to begin its first observing run (designated O1) in September of 2015. Prior to this observing run, extensive work was underway to develop optimized observation strategies for the detection and characterization of kilonovae. This includes comprehensive studies into the impact of contamination on wide-field searches and how it can be mitigated. These studies involved looking at both simulated observations and deriving empirical contamination rates from observations designed to be representative of real gravitational wave follow-up \citep{CowpBerger15,Cowp+17b}. This work is discussed in detail in \cref{ch:ch2,ch:ch3}.

\subsection{GW150914: The First Direct Detection of Gravitational Waves}
\label{sec:intro_gw150914}
The first direct detection of gravitational waves by Advanced LIGO occurred on 2015 September 14, when the instrument was in an engineering configuration, just days before the O1 official start date of 2015 September 18 \citep{LIGOGW150914}. The gravitational  wave signal was found to be consistent with expectations for the merger and ringdown of a binary black hole system with component masses of $36+29~\msun$ at a luminosity distance of $D_L \approx 410$~Mpc \citep{LIGOGW150914}. This provided the first definitive evidence that such systems exist and merge.

While the merger of a binary black hole system was not expected to produce any EM counterpart, there was nevertheless a large-scale effort to observe the localization region. This effort consisted of more than 20 teams, conducting observations from radio to $\gamma$-rays, along with searches for coincident neutrino emission \citep{LIGOGW150914FollowUp,LIGOGW150914FollowUpSupp,GW150914IceCube,Annis+16,GW150914Fermi,Evans+16,Kasliwal+16,Savchenko+16,GW150914PS1,GW150914DECam,Tavani+16}. This included comprehensive optical follow-up by our team using DECam to cover \apx100~deg$^2$ (\apx40~(11)\% of the initial~(final) localization probability) to a limiting magnitude of $i \apx 22.5$~mag, some of the deepest follow-up conducted in response to GW150914. While these searches did not identify a robust counterpart, there were initial reports of a coincident short and weak GRB detected by the \fermi/GBM detector, just 0.4s after the GW event \citep{GW150914Fermi}. However, this event was not detected in analysis of the {\it INTEGRAL} $\gamma$-ray data \citep{Savchenko+16} and was also not seen in an independent re-analysis of the \fermi/GBM data \citep{Greiner+16}.

\subsection{Continued Detections of Binary Black Holes}
\label{sec:intro_bbhs}
LIGO would see continued success during the rest of O1. On 2015 December 26, LIGO detected the second high-significance GW signal from the inspiral and merger of two black holes. The signal, named GW151226, was consistent with the merger of $14+8~\msun$ black holes at a luminosity distance of $D_L \approx 440$~Mpc. This event again prompted a large follow-up effort \citep[see e.g.,][]{GW150914PS1,Yoshida+17}, including deep optical observations by our DECam program \citep{Cowp+16}. Again, no credible electromagnetic counterparts were found by any follow-up program. The optical follow-up performed in response to GW151226 is discussed in detail in \cref{ch:ch4}.

The Advanced LIGO and Virgo collaboration has since announced the detection of a further three GW signals, all consistent with the inspiral and merger of a binary black hole system. These events are GW170104 \citep{LIGOGW170104}, GW170608 \citep{LIGOGW170608}, and GW170814 \citep{LIGOGW170814}. The event GW170104 is noteworthy as it is the most distant detection to date at a luminosity distance of $D_L \approx 880$~Mpc. The event GW170814 is also important as it was the first BBH merger detected by both Advanced LIGO and Virgo. This event highlights the remarkable improvement in localization that can be achieved in a three detector network as the inclusion of the Virgo detection reduces the 90\% credible region from 1160 deg$^2$ to 60 deg$^2$ \citep{LIGOGW170814}. The continued and frequent detection of binary black hole systems suggests that they will likely dominate the detection rates compared to BNS or NS-BH binaries \citep{LIGOBBHRates}.

\subsection{GW170817: The Dawn of Joint GW-EM Astronomy}
\label{sec:intro_gw170817}
The Advanced LIGO and Virgo collaborations made the first detection of a GW signal consistent with the inspiral and merger of two neutron stars \citep{LIGOGW170817} on 2018 August 17. The source, named GW170817, was found to be remarkably close at a luminosity distance of just $D_L \approx 40$~Mpc. Even more remarkable was the detection of coincident $\gamma$-rays, seen by both the \fermi/GBM and {\it INTEGRAL} instruments just 1.8~s after the GW detection \citep[GRB\,170817A;][]{LIGOGW170817grb,GW170817Fermi,Savchenko+17}. Thus, with the combined detection of gravitational waves and electromagnetic radiation, GW170817 heralded the dawn of true joint GW-EM Astronomy.

Less that twelve hours after the initial announcement by LIGO, a massive observational campaign began in an attempt to identify a counterpart other than GRB\,170817A. This goal was quickly achieved as several teams, including our DECam effort, found an optical counterpart in the nearby galaxy NGC4993 \citep{LIGOMMAPaper,Arcavi+17,Coulter+17,GW170817DECam,Valenti+17}. Multi-wavelength observations have also led to the discovery of an associated radio and X-ray source \citep{Alexander+17,Davanzo+18,Margutti+17,Margutti+18,Mooley+18,Troja+17}, detailed studies of the host galaxy \citep{Blanchard+17,Cantiello+18}, and confirmation of the notion that compact object mergers are the progenitors of SGRBs \citep{Fong+17}.

In the context of this thesis, the most relevant work is that surrounding the optical emission associated with GW170817. Numerous independent efforts to model this optical emission have all concluded that it is consistent with the expectations for a kilonova \citep{Cowp+17,Kilpatrick+17,Tanaka+17,Villar+17b, Tanaka+18}. A detailed discussion of our initial observations and modeling efforts can be found in \cref{ch:ch5}. There are two key points derived from this analysis. First, the optical emission is best described by a multi-component kilonovae model with a total ejecta mass of $\Mej \approx 0.05~\msun$. The multiple components suggest that the ejecta contains material that is both lanthanide-rich and lanthanide-poor. Second, the large inferred ejecta mass suggests that BNS mergers, like GW170817, are a dominant site for the cosmic production of heavy $r$-process elements.

The third LIGO observing run (O3) is expected to begin in November or December of 2018. During this time the LIGO and Virgo instruments will continue to improve in sensitivity, probing more and more distance mergers. It is expected that many more BNS mergers will be detected during O3 and beyond. Continued success at the level of GW170817 requires efficient and dedicated use with the next-generation of observational facilities. \cref{ch:ch6} presents a discussion on the prospects of detecting kilonovae with the Large Synoptic Survey Telescope \citep[LSST;][]{Ivezic+09}, the premiere time-domain instrument of the next decade. We conclude with the notion that LSST can and should play a major role in GW follow-up, but only if a dedicated target-of-opportunity program is put in place.
