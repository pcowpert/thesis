%No preamble material
Grandiose intro paragraph about joint GW-EM astronomy

Top level description of thesis 

Outline of this chapter.

\section{Gravitational Waves}
\label{sec:intro_gw}
Brief history of gravitational waves: GR, Einstein's Paper, Basic idea of what they are.

Very brief mathematical introduction

Argument that they must be quadrupole and resulting polarization

Something about translating to radiation and $1/r$ relation

Intuition about strain scales

\subsection{GW Sources}
\label{sec:intro_gw_source}
EM Radiation is accelerating charges. GW is accelerating mass. Small strain scales mean that we have to consider drastic accelerations of a lot of mass

1-2 paragraphs on CMB, background, SMBH, etc.

Relevant to this thesis is compact object binary mergers

\subsection{GW Detectors}
\label{sec:intro_gw_det}
GW radiation is nice because it does not disperse but makes them hard to detect

A brief paragraph about the early bar measurements, HT Pulsar, focus on context of thesis

1-2 paragraphs on ground-based detectors, functionality, and sensitivity 

\section{Joint GW-EM Astronomy}
\label{sec:intro_joint}
Talk about why we want to do joint GW-EM astronomy. Can do a lot of cool physics with GW, but Gw-EM is better. What are the science gains and limitations of individual detections vs. Joint

\subsection{EM Counterparts}
\label{sec:intro_counterparts}
Important to understand the range of possible emission

GRB

On-Axis Jet

Off-Axis Jet

Kilonova (1-2 paragraphs)

Precursors (Neutron/Cocoon)

Late-time radio emission.

\subsection{Search Methods}
\label{sec:intro_searchmethods}
Once you know what you are looking for, need to understand how to find it. 

Major difficulty is large localization regions 

\subsubsection{Galaxy-Targeted}
\label{sec:intro_galaxy}
General idea of galaxy targeted searches

Pros - Available to telescopes with small FoV, in small localization regions potentially quicker than wide-field.

Cons - Localization and distance errors matter. Can quickly encounter more galaxies than doable per night

\subsubsection{Wide-Field Searches}
\label{sec:intro_widefield}
General idea of wide field searches and instruments involved

Pros - Cover whole region quickly with little bias, place limits on non-detections

Cons - Contamination and time available

\section{Detections with Advanced LIGO/Virgo}
\label{sec:intro_history}
Advanced LIGO time line and lead up to first observing run 

\subsection{GW150914: The First Detection}
\label{sec:intro_gw150914}
The GW detection of GW150914. First BBH. Surprisingly massive.

Massive EM follow-up campaign.

The \fermi\ stuff

Ushered in exciting new era

\subsection{Binary Black Holes. So Many.}
LIGO would have continued success during O1 detecting a second BBH merger and strong candidate

GW170104 - The most distance detection

GW170814 - Virgo joining LIGO, the first three detector BBH merger, massive effect on localization 

\subsection{GW170817: The Dawn of Joint GW-EM Astronomy}
The detection of GW170817. GW information and localization due to VIRGO

The GRB. 

Optical counterpart and observations. 

Radio and X-Ray stuff.

Slightly more detailed outline of thesis compared to preamble

%Don't include any calls to \bibliography
