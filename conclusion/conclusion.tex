% No Preamble Material
This thesis presents a series of studies, both theoretical and observational, that track the emerging field of joint GW-EM astronomy leading up to the detection of GW170817, the first binary neutron star merger and associated EM counterpart. GW170817 is a truly remarkable event as it singularly identifies the point in time at which joint GW-EM astronomy transitioned from a hopeful idea to a newly birthed field of astronomy. Therefore, I will conclude this thesis with a brief discussion covering the implications of GW170817. This includes both understanding the major results from the discovery and optical/NIR observations, along with addressing some key considerations as the field looks ahead to future detections.

Broadly speaking, the observed properties of GW170817 represent a resounding success for the theory of kilonovae developed in the years leading up to the first discovery \citep{LP98,Rosswog+99,Metzger+10,BarnesKasen13,FernandezMetzger13,TanakaHotokezaka13,Barnes+16,Sekiguchi+16}. The observed timescale and luminosity, along with the detection of multiple ejecta components and their associated masses and velocities, are well in line with the predictions and simulations developed over the last decade. As a result, this event has offered unprecedented insight into the merger, the dynamics and energetics of the system, as well as the progenitors and ultimate remnant. Specifically, from the optical and NIR data gathered for this event alone we can say the following:
\begin{itemize}
\item {\bf Breaking Modeling Degeneracies:} The presence of a bright blue emission component during the first few days is indicative of either material ejected dynamically at the collision interface \citep{Nicholl+17a} or neutrino-heated winds from a short-lived magnetar \citep{Metzger+18}. In either case, this suggests that the merger was of two neutron stars and the ultimate remnant was a stellar-mass black hole. Thus, electromagnetic observations are able to provide insight into the system that was otherwise ambiguous from the GW data alone.
\item {\bf Cosmic Nucleosynthesis:} Both the uniquely red photometry \citep{Cowp+17} and broad features identified in the NIR spectroscopy \citep{Chornock+17} provide the first direct evidence that $r$-process nucleosynthesis occurs in BNS mergers, a longstanding question in astrophysics \citep{LattimerSchramm74,Lattimer+77}. Furthermore, the merger rate derived by LIGO \citep{LIGOGW170817} combined with the inferred mass and composition of the ejected mass \citep{Cowp+17,Villar+17b}, are suggestive that BNS mergers can be a dominant source of $r$-process element production in the Universe.
\item {\bf Multiple Ejecta Components:} The presence of both red and blue emission from GW170817 is suggestive of several distinct ejecta components each with their own physical mechanisms and nucleosynthetic yields. Thus, we can conclude that not only can BNS mergers explain the observed Galactic $r$-process yields, they may also explain the observed ratio of light $r$-process $(A \lesssim 140)$ and heavy $r$-process elements $(A \gtrsim 140)$ \citep{Cowp+17,Kasen+17}.
\item {\bf Implications for NS Equation of State:} The amount of material ejected during the merger depends sensitively on the NS equation of state. The GW detection provides an upper bound on the measured NS tidal deformations. Complimentary to this, the EM detections provide a lower bound on the tidal deformation \citep{Radice+18}. Thus, we are able to place limits on both the extremely stiff and extremely soft equations of state. This is not possible from GW or EM data alone.
\item {\bf Joint GW-EM Cosmology:} The Hubble Constant $(H_0)$ can be computed from the joint detection by comparing the luminosity distance determined from the gravitational wave data and the redshift of the host galaxy from EM observations. For GW170817, this leads to a measurement of $H_0 = 70^{+12}_{-8}$ km s$^{-1}$ Mpc$^{-1}$ \citep{LIGOH0}. Including additional constraints from the X-ray/Radio observations gives a value of $H_0 = 75.5^{+11.6}_{-9.6}$ km s$^{-1}$ Mpc$^{-1}$ \citep{Guidorzi+17}. This provides a unique way to make measurements of local cosmological parameters that is independent of other methods (e.g., Type Ia SNe, CMB, BAOs) and independent of the cosmic distance ladder \citep{Schutz1986,HolzHughes05}.
\end{itemize}

This treasure trove of information serves as a testament to the immense insight offered by joint GW-EM detections. However, GW170817 was just a singular event, and like any new field of study there are major open questions that remain. A few key open questions are:
\begin{itemize}
\item {\bf Ubiquity of Behavior:} The most important question to be answered by future detections is accessing the uniqueness of the emission observed from GW170817. Are all BNS mergers so initially bright and blue or is this emission only viewable from certain lines of sight? Do we always see multiple ejecta components and what are the ranges of ejecta masses and opacities? All of these questions have important implications for the detectability of mergers as well as improving our understanding of neutron stars, their equation of state, and nucleosynthesis yields.
\item {\bf Prospects for NS-BH Mergers:} Yet to be detected by LIGO is the merger of a neutron star and a black hole. In this scenario the lack of a collision interface or long-lived HMNS remnant means that the ejecta will be dominated by tidal tail material and accretion disk winds. As a result, the kilonova is unlikely to be as initially bright and blue. This combined with the likelihood that such events will be detected much further away $(\gtrsim 200$~Mpc) means that the detection of such events will be difficult, requiring large-aperture telescopes (e.g., DECam, LSST).
\item {\bf The Future of GW-EM Cosmology:} Continued refinement of the $H_0$ measurement from GW-EM astronomy requires continued detection of events and their counterparts. To achieve a precision on par with other methods (e.g., $\sim1\%$) will require the detection of an additional$50-75$ events \citep{Guidorzi+17}. It will also be important to consider, as improved sensitivity allows the detection of more distance systems, if one can move beyond just local measurements of $H_0$ to constraining other cosmological parameters such as the $w$ parameter for dark energy. 
\end{itemize}

These open questions are vast and like any scientific endeavor they will require time to address fully. The third Advanced LIGO/Virgo observing run (O3, Scheduled $\sim$2018-2019) will extend our detection sensitivity out to $\gtrsim100$~Mpc for BNS mergers and even further for NS-BH mergers, allowing the detection of even more distance sources. Additional detectors coming online in the next decade, such as KAGRA \citep{KAGRA} and LIGO-India \citep{LIGOIndia}, will allow us to better localize sources making searches for EM counterparts potentially less difficult. This improvement combined with next generation EM instruments like LSST, JWST, and WFIRST will massively accelerate the kind of science we can accomplish with GW-EM multi-messenger observations. Ultimately, the most powerful aspect of joint GW-EM astronomy will be showing that the open questions presented here are {\it answerable.}

% https://youtu.be/5sYtRpnOJzU?t=2401
% https://arxiv.org/pdf/1710.05931.pdf
% https://ui.adsabs.harvard.edu/?#abs/2017arXiv171005931M/references